\documentclass[a4paper,11pt]{article}
\begin {document}
{
\huge
Symmetry -- Group Theory Library
}

\tableofcontents

\section{Introduction}
	Yet another abstract algebra library. What is different from existing ones (GAP,MAPLE,SINGULAR)? Well, the main differences are the focus only on Group Theory and the fact that this one is actually maintainable, written entirely in C++ with a nice object oriented design. Other unique trait is the user interface which is provided in a webpage by a minimal custom http server. Besides the usual operations provided by an abstract algebra library(create a group and investigate it's elements, properties, subgroups, isomorphisms, etc) the user will be able to actually visualize the group structure through the geometry it describes by exploiting the relation between group theory and geometry. These drawings are made with the help of Java Script and WebGl technology and rendered with the graphics card(using acceleration).
	So, the application is logically composed from the following components:
\begin{enumerate}
\item
Algebra Library -- written entirely in C++ 0X (using STL and BOOST) with a generic design made possible by template programming.
\item
A custom HTTP server -- used to serve the user interface(html pages) and written also in C++ with BOOST ASIO.
\item
A Java Script(WebGl) engine -- used to render 3d graphics and make the user interation easier.
\item
A C++ distributed engine -- which connects the user interface with the algebra library, by spawning multiple processes on (different) machines, gathering and feeding data to the HTTP server
\end{enumerate}

\section{Components}

\subsection{The library}
\section{Development}
\end {document}
